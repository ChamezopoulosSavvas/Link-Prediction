\begin{abstract}

Η ανάπτυξη της επιστημονικής μελέτης της  εξέλιξης των δικτύων έχει φέρει μια νέα εποχή στην επίλυση διαφόρων πρακτικών προβλημάτων. Παράλληλα, οι τεχνικές μηχανικής μάθησης, αλλα και γενικότερα της τεχνητής νοημοσύνης,
επιταχύνουν και βελτιώνουν τον τρόπο που η πληροφορία επεξεργάζεται, διευκολύνοντας και επιταχύνοντας
σημαντικά την επίλυση υπερβολικά χρονοβόρων ή και άλυτων μέχρι σήμερα προβλημάτων. Η τελευταία καινοτομία
στο χώρο της θεωρίας δικτύων και της ανάλυσης φαινομένων με χρήση τεχνητής νοημοσύνης υποστηρίζει το συνδυασμό των 
δύο, δηλαδή επίλυσης προβλημάτων δικτύων με χρήση τεχνικών μηχανικής μάθησης. Οι εφαρμογές της παραπάνω προσέγγισης βρίσκονται
σε πολυάριθμους τομείς.  Στα πλαίσια της εργασίας αυτής επιλέχθηκαν δύο περιπτώσεις προς ανάλυση από διαφορετικούς τομείς. Η πρώτη ασχολείται με ένα δίκτυο αναφορών σε  επιστημονικές δημοσιεύσεις ενώ η δεύτερη με ένα δίκτυο που αφορά την εμπιστοσύνη μεταξύ χρηστών σε πλατφόρμα
συναλλαγών κρυπτονομισμάτων. Σκοπός της εργασίας, είναι να προβλεφθούν, στην πρώτη περίπτωση, οι νέες αναφορές μεταξύ των
επιστημονικών εργασιών στο δοθέν δίκτυο και, στη δεύτερη περίπτωση,  η κίνηση της αγοράς πάνω σε μία από τις γνωστότερες πλατφόρμες συναλλαγών
κρυπτονομισμάτων. Για το σκοπό αυτό χρησιμοποιούμε δεδομένα γράφων σε συνδυασμό με μια πληθώρα τεχνικών 
μηχανικής μάθησης, με ένα νέο συνδυαστικό τρόπο, στην αναζήτηση της αποδοτικότερης λύσης με σκοπό την
ανάπτυξη ενός ολοκληρωμένου συστήματος πρόβλεψης συνδέσεων σε γράφους.
 
   \begin{keywords}

   Πρόβλεψη Συνδέσμων, Λογιστική Παλινδρόμιση, Βεβαρημένα Δίκτυα, Αναπαράσταση Κόμβων, 
   \tl{Node2Vec, GraphSAGE, CTDNE} 
   \end{keywords}
\end{abstract}



\begin{abstracteng}
\tl{The recent advances on the scientific examination of the evolvement of  networks has improved dramatically the solution techniques regarding various practical problems. At the same 
time, machine learning, a subcategory of artificial intelligence algorithms and techniques, has been used
more and more in solving complex and previously time-consuming and/or unsolvable tasks. The latest trend
in the field of network theory and machine learning has the two combined, in order to achieve 
unprecedented results. The applications of the aforementioned approach span many different areas. In this thesis, two cases from different areas are examined. The first one  concerns a citation network of scientific publications while the second one concerns a trust network among users of a cryptocurrency platform.
The main aim of this work, in the first case, is to predict future citations in the given network of publications  and, in the second case, to predict the future state of  a network consisting of users of
a well-known cryptocurrency platform. To achieve that, these cases are viewed as link prediction problems and we utilize graph data available for these networks and use various machine learning techniques in order to provide the most efficient solution.}

   \begin{keywordseng}
       \noindent
   \tl{ Link Prediction, Logistic Regression, Weighted Networks, Node Embedding, Node2Vec, GraphSAGE, CTDNE}
   \end{keywordseng}

\end{abstracteng}