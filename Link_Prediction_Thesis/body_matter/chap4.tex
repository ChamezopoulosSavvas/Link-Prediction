\chapter{Συμπεράσματα και Μελλοντικές Επεκτάσεις}

\section{Συμπεράσματα}

Σύμφωνα με την ανάλυση που έγινε, μπορεί να εξαχθεί ένας αριθμός συμπερασμάτων. Αρχικά, όπως είναι 
αναμενόμενο η κάθε τεχνική λειτουργεί καλύτερα στο σύνολο δεδομένων που ταιριάζει περισσότερο στις
σχεδιαστικές της ανάγκες. Το \en{CORA dataset} μιας και περιέχει πληθώρα χαρακτηριστικών κόμβων λειτουργεί
καλύτερα με την τεχνική \en{GraphSAGE}, ενώ η \en{CTDNE} λειτουργεί βέλτιστα όπου υπάρχει διαθέσιμη η 
χρονική παράμετρος.

Η τεχνική \en{Node2Vec} από την άλλη παρουσιάζει πολύ ικανοποιητική απόδοση και στις δύο περιπτώσεις.
Αυτό ενδεχομένως να οφείλεται στο σχεδιασμό της, που την κάνει να χρησιμοποιεί αποκλειστικά τα δομικά
χαρακτηριστικά του γράφου και όχι κάποια άλλη παράμετρο (όπως χαρακτηριστικά κόμβων ή τη χρονική 
παράμετρο).

Κάτι που διαφαίνεται στα αποτελέσματα είναι η σημαντικότητα της βελτιστοποίησης των παραμέτρων. Όπως
αναφέρθηκε, σε κάθε τεχνική και πείραμα εκτελούνται σε διάφορα σημεία ρουτίνες εύρεσης βέλτιστων
παραμέτρων και τελεστών. Στον αντίποδα, όταν χρησιμοποιούνται υποβέλτιστες παράμετροι για την 
εφαρμογή μιας τεχνικής ή βέλτιστες παράμετροι άλλης τεχνικής, η απόδοση πέφτει αρκετά, όπως φαίνεται 
στην περίπτωση του πειράματος με χρήση \en{Node2Vec} με τις βέλτιστες παραμέτρους για \en{CTDNE}.

\subsection{Φυσική ερμηνεία αποτελεσμάτων}

Όπως αναλύθηκε και στην εισαγωγική Ενότητα \ref{problem_statement}, ουσιαστικά γίνεται μια προσπάθεια
να προβλεφθεί η μελλοντική εικόνα του δικτύου. Αυτή η λειτουργία έχει διαφορετική φυσική ερμηνεία,
ανάλογα με τα δεδομένα πάνω στα οποία εφαρμόζεται και το πρόβλημα το οποίο μοντελοποιεί το εκάστοτε 
δίκτυο.

Στην περίπτωση του \en{CORA Dataset}, η φυσική ερμηνεία του προβλήματος αφορά την πρόβλεψη  των μελλοντικών αναφορών στο παραπάνω  δίκτυο επιστημονικών εργασιών. Θα πρέπει να σημειωθεί ότι κατα τη μελέτη οι συνδέσεις στο εν λόγω 
πρόβλημα θεωρήθηκαν ως  μη κατευθυνόμενες.
Επίσης, αφού οι ακμές στο εν λόγω σύνολο δεδομένων δεν έχουν βάρη, θεωρούνται ισότιμες. Η κάθε 
επιστημονική δημοσίευση (κόμβος), όπως και στην πραγματικότητα, είναι ισότιμη με οποιαδήποτε άλλη,
και η ````σημαντικότητα'''' αυτών συνδέεται με την έννοια της ````δημοτικότητας''''. Δηλαδή αυτές που αναμένεται 
να χρησιμοποιηθούν περισσότερο στο μέλλον, είναι κατα τη γενική περίπτωση οι δημοσιεύσεις που 
ήδη έχουν περισσότερες αναφορές. Εξ' ορισμόυ καμία 
επιστημονική δημοσίευση δεν μπορεί να θεωρηθεί σημαντικότερη από κάποια άλλη. Η σημαντικότητα συνήθως εξαρτάται από τη δημοτικότητα που έχει. Έτσι, υπάρχουν κάποιες
δημοσιεύσεις που χρησιμοποιούνται από την επιστημονική κοινότητα (για τον οποιονδήποτε λόγο) περισσότερο από κάποιες άλλες 
και έτσι μπορεί να θεωρηθεί ότι έχουν μεγαλύτερο επιστημονικό αντίκτυπο.

Από την άλλη, στην περίπτωση του \en{OTC Bitcoin Dataset}, η φυσική ερμηνεία του προβλήματος και των
αποτελεσμάτων έχει μια εντελώς διαφορετική διάσταση, μιας και το σύνολο δεδομένων αντιπροσωπεύει ένα 
εντελώς διαφορετικό περιβάλλον. Όπως αναλύθηκε στην σχετική Ενότητα (\ref{BTCDataset}), το σύνολο
δεδομένων ουσιαστικά αντιπροσωπεύει μια αγορά όπου το μοναδικό προϊόν προς ανταλλαγή είναι το 
\en{Bitcoin}. Στην περίπτωση αυτή, οι ακμές αντιπροσωπεύουν βαθμολογίες από χρήστη σε χρήστη. 
Σκοπός της όλης διαδικασίας, όπως αναφέρεται και στη σχετική δημοσίευση \cite{kumar2018rev2}, είναι
ο εντοπισμός των ψευδών ή/και των αναξιόπιστων χρηστών. Η παραδοχή είναι πως ένας χρήστης με χαμηλή 
βαθμολογία, θα 
θεωρείται είτε ψευδής, είτε αναξιόπιστος, και επομένως η δημοτικότητά του θα φθίνει.

Η βαθμολογία αυτή φυσικά περιέχει και ένα βάρος, το οποίο στην αρχική του μορφή ήταν μία ακέραια τιμή
(που ανήκει στο  \( [-10,10] - \{0\}\)). Συνεπώς το αρχικό σύνολο δεδομένων αφορά ένα κατευθυνόμενο και προσημασμένο
δίκτυο. Οι τεχνικές που χρησιμοποιήθηκαν ωστόσο αδυνατούν να επεξεργαστούν προσημασμένα (\en{signed})
δίκτυα, όποτε τα βάρη των ακμών (δηλ. οι τιμές των βαθμολογιών) προσαρμόστηκαν έτσι ώστε οι τιμές τους
να ανήκουν στο \( [0,1]\). Έτσι η πρόβλεψη έγινε με την υπόθεση πως ένας χρήστης (κόμβος) αν έχει ακμές που 
καταλήγουν σε αυτόν με μικρό ή μηδενικό βάρος, τότε θα υπάρχει μια τάση στο μέλλον αυτές οι ακμές να 
φθίνουν περαιτέρω σε αριθμό και βάρος, μέχρι στο τέλος να μηδενιστούν. Αυτό σημαίνει ότι αν
ένας χρήστης για οποιονδήποτε λόγο αποκτά χαμηλή βαθμολογία, τότε όλο και λιγότεροι χρήστες
θα συναλλάσσονται με αυτόν, βάζοντάς του παράλληλα χαμηλές βαθμολογίες, με κατάληξη να μη συναλλάσσεται
κανένας χρήστης με αυτόν (απορρίπτοντάς τον τελικά από το σύνολο και την αγορά). Στον αντίποδα, ένας χρήστης
που παρουσιάζει συνέπεια στις συναλλαγές του, θα τείνει να βαθμολογείται με υψηλότερες βαθμολογίες
από όλο και περισσότερους χρήστες. Αυτό στο μέλλον θα προσελκύει όλο και περισσότερους χρήστες, 
και όσο ο εν λόγω χρήστης διατηρεί τη συνέπειά του, η βαθμολογία του θα έχει αύξουσα τάση.

Φυσικά και οι δύο περιπτώσεις αφορούν μια αυστηρή μαθηματική προσέγγιση. Στον πραγματικό κόσμο 
εμπλέκονται και διάφοροι άλλοι παράγοντες όπως ο ανθρώπινος παράγοντας, το γενικότερο οικονομικό και 
κοινωνικό περιβάλλον, αλλά και  κάποιοι ειδικοί παράγοντες που αφορούν το κάθε σύνολο δεδομένων.
Για παράδειγμα, στο \en{CORA Dataset} ενδέχεται ένα έργο να μην αναφέρεται από πολλούς, χωρίς αυτό να 
λέει κάτι για την ποιότητά του. Μπορεί να αφορά ένα μικρό ή/και νέο επιστημονικό πεδίο, είτε να
αναλύει μια πολύ εξειδικευμένη δουλεία που αφορά μια πολύ συγκεκριμένη μερίδα του επιστημονικού χώρου.
Συνεπώς, μια φαινομενικά αντιδημοτική δημοσίευση μπορεί είτε να αφορά πολύ μικρό κοινό, είτε να
αφορά ένα νέο επιστημονικό πεδίο.

Σχετικά με το \en{OTC Bitcoin Dataset}, το θέμα της βαθμολογίας ενός χρήστη είναι 
πιο σύνθετο. Ένας χρήστης ενδεχομένως να αποκτά χαμηλή βαθμολογία όχι λόγω κακόβουλης διαχείρισης των
πόρων του, αλλά λόγω απειρίας ή απλώς λαθών στις συναλλαγές του. Συνεπώς η βελτίωση όσο και η μείωση της
βαθμολογίας είναι κάτι που εξαρτάται μεν από την υπόλοιπη κοινότητα, μπορεί όμως δε να επηρεαστεί 
σημαντικά και από τον ίδιο το χρήστη τον οποίο αφορά.


\section{Μελλοντικές επεκτάσεις}

Στα πλαίσια των μελλοντικών επεκτάσεων, θα μπορούσαν αυτές οι τεχνικές  να εφαρμόστούν σε
περισσότερα σύνολα δεδομένων ώστε να διερευνηθεί η αποδοτικότητα του συστήματος πάνω σε διαφορετικές
εφαρμογές. Η λίστα ζητημάτων που έχουν μοντελοποιηθεί ως προβλήματα πρόβλεψης σε γράφους είναι πολύ μεγάλη
και συνεχώς προστέιθενται νεες εφαρμογές σε αυτή. Μπορούν να εξερευνηθούν προβλήματα με διαφορετικές
ιδιότητες (κατευθυνόμενοι, προσημασμένοι, γράφοι με/χωρίς βάρη, με/χωρίς χαρακτηριστικά κόμβων, ακμών,
κτλ.)

Ακόμα, θα μπορούσαν να εφαρμοστούν διαφορετικές τεχνικές πάνω στα ίδια δεδομένα, με σκοπό
την εύρεση αυτής της τεχνικής εξαγωγής αναπαραστάσεων που λειτουργεί βέλτιστα για τα συγκεκριμένα
δεδομένα. Στη βιβλιογραφία υπάρχουν αρκετές τεχνικές που έχουν αναπτυχθεί, που είτε αφορούν γράφους με 
συγκεκριμένες ιδιότητες, είτε εξυπηρετούν την επίλυση προβλημάτων συγκεκριμένης φύσεως. Αυτό δε σημαίνει
πως δεν μπορούν να συνδυαστούν οι οποιεσδήποτε τεχνικές και δεδομένα ώστε να συνθέσει ένα
εντελώς νεο σύστημα πρόβλεψης. Άλλωστε, κατι ανάλογο έγινε και στα πλαίσια τη εργασίας αυτής κατα
την χρήση της τεχνικής \en{GraphSAGE} στα δεδομένα \en{OTC Bitcoin} με χαρακτηριστικά κόμβων που 
κατασκευάστηκαν με χρήση της τεχνικής εξαγωγής αναπαραστάσεων κόμβων \en{Node2Vec}!

Τέλος,  θα μπορούσαν να αναζητήθούν οι βέλτιστες σχεδιαστικές παραμέτροι σε ακόμα περισσότερα
σημεία, όπως ο βέλτιστος διαχωρισμός των δεδομένων, η βέλτιστη επιλογή ταξινομητή και παραμέτρων αυτού,
οι βέλτιστες παράμετροι των τεχνικών εξαγωγής αναπαραστάσεων, κ.α. Οι επιλογές είναι πρακτικά πάρα πολλές, 
και αυτή η εργασία απλώς ````έξυσε την επιφάνεια'''' του προβλήματος. Ο αναγνώστης παροτρύνεται να πάρει 
αυτή την εργασία και να την εξελίξει με όποιον τρόπο μπορεί. Άλλωστε ο σκοπός κάθε επιστήμονα και 
μηχανικού είναι η εύρεση μιας λύσης καλύτερης από την βέλτιστη υπάρχουσα, με σκοπό τη βελτίωση της λύσης
κάθε προβλήματος. Αυτη είναι και η φιλοσοφία που έχει οδηγήσει το ανθρώπινο είδος στην άνθηση, τον
πολιτισμό και την ευημερία.