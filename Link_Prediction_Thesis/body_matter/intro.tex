\chapter{Εισαγωγή}

Η έννοια των δικτύων υπήρχε από πολύ παλιά τόσο στη ζωή και στη φύση, όσο και στην κοινωνία των ανθρώπων.
Από τις συνδέσεις των ατόμων που σχηματίζουν μόρια, μέχρι τις πόλεις και το ίδιο το διαδίκτυο, παντού
τριγύρω μας μπορούμε να διακρίνουμε συμπλέγματα οντοτήτων που αλληλοεπιδρούν αναμεταξύ τους. Με την 
ανάπτυξη των υπολογιστών και των υπολογιστικών συστημάτων, καθώς και με την πρόοδο των μαθηματικών
έγινε δυνατή η ανάλυση και η εξαγωγή πληροφορίας και γνώσης που κρύβουν τα δίκτυα μέσα τους. Παράλληλα,
η ραγδαία ανάπτυξη της τεχνητής νοημοσύνης και ιδιαίτερα των τεχνικών μηχανικής μάθησης, έφερε μια νέα
εποχή στην ανάλυση και επίλυση μιας πληθώρας προβλημάτων, καθιστώντας επιλύσιμα πολύ περίπλοκα ζητήματα
με υψηλή ακρίβεια και σε σχετικά μικρό χρόνο.

Πολλά προβλήματα της καθημερινής ζωής είναι δυνατόν να λυθούν με χρήση θεωρίας δικτύων και χρήση 
υπολογιστικών συστημάτων. Μερικά από τα πιο γνωστά είναι:

\begin{itemize}
    \item Ταξινόμηση Κόμβων
    \item Ομαδοποίηση Κόμβων
    \item Πρόβλεψη συνδέσεων
    \item Οπτικοποίηση Δικτύου
    \item και άλλα (κ.α.)
\end{itemize}

Στην εργασία αυτή θα ασχοληθούμε με το πρόβλημα της πρόβλεψης συνδέσεων σε  γράφους, οι οποίοι αποτελούν μια μορφή μοντελοποίησης
 των δικτύων. Το πρόβλημα  την πρόβλεψης συνδέσεων συναντάται συχνότατα τόσο στην καθημερινή ζωή,
όσο και σε πιο εξειδικευμένες περιπτώσεις τόσο στην κοινωνία όσο και στην οικονομία και τη φύση.
Ένα πολύ διαισθητικά κατανοητό παράδειγμα βρίσκεται στα μέσα κοινωνικής δικτύωσης \cite{al2011survey}. 
Οι εταιρίες πίσω 
από τις εν λόγω πλατφόρμες, προκειμένου να διατηρούν τους χρήστες ενεργούς, καλούνται να τους
προτείνουν να συνδεθούν με άτομα που πιθανόν να γνωρίζουν, να τους προτείνουν υλικό και σελίδες που
πιθανόν τους ενδιαφέρουν κ.α. Από την άλλη, στις πλατφόρμες διαδικτυακών αγορών, προτείνονται στους
χρήστες προϊόντα προς πώληση που πιθανόν τους ενδιαφέρουν, βάσει του δικού τους ιστορικού  ή  άλλων
χρηστών.
Ακόμα και σε θέματα ασφάλειας έχουν υλοποιηθεί εργαλεία με χρήση της πρόβλεψης συνδέσεων 
\cite{berlusconi2016link}. Τέλος, μια εφαρμογή του προβλήματος της πρόβλεψης συνδέσεων στην οικονομία
και πιο συγκεκριμένα στο χρηματιστήριο είναι η πρόβλεψη της κίνησης της αγοράς, με την έννοια της
πρόβλεψης των τασεων της αγοράς και της αυξομοίωσης της ζήτησης των μετοχών 
\cite{stockpred}, \cite{Souza_2019}.

Συγκεκριμένα, στην εργασία αυτή θα ασχοληθούμε με δύο εφαρμογές του προβλήματος της πρόβλεψης συνδέσμων.
Αρχικά, θα προσπαθήσουμε να προβλέψουμε τις συνεργασίες μεταξύ επιστημόνων στο μέλλον, χρησιμοποιώντας
δεδομένα αναφορών από έναν επιστήμονα σε κάποιον άλλον στις δημοσιεύσεις τους. Επιπλέον, θα 
προσπαθήσουμε να προβλέψουμε την κίνηση της αγοράς του κρυπτονομίσματος \en{Bitcoin} μέσω της 
πλατφόρμας \en{OTC}. Για τον σκοπό αυτό, θα χρησιμοποιήσουμε δεδομένα βαθμολόγησης μεταξύ των χρηστών,
και θεωρούμε πως ένας χρήστης με υψηλή βαθμολογία θα εκτελεί περισσότερες συναλλαγές στο μέλλον, 
αφού οι υπόλοιποι θα τον προτιμούν, ενώ ένας χρήστης με χαμηλότερη βαθμολογία σταδιακά θα πάψει
να συναλλάσσεται αφού δε θα υπάρχουν ενδιαφερόμενοι να αγοράσουν από ή να πουλήσουν σε αυτόν.

\section{Διατύπωση του προβλήματος} \label{problem_statement}

Το πρόβλημα που θα αναλυθεί στην εργασία αυτή διατυπώνεται ως εξής:
Δεδομένου ενός γράφου που μεταβάλλεται χρονικά, θα γίνει μια προσπάθεια πρόβλεψης των μελλοντικών συνδέσεων. Αυτή η δήλωση 
ισοδυναμεί με τη σκέψη πως αν έχουμε διαθέσιμη μια εικόνα του δικτύου κάποια στιγμή στο παρελθόν, 
τότε θα προσπαθήσουμε να διατυπώσουμε την εικόνα του γράφου αυτού στο παρόν, προβλέποντας τις νέες 
συνδέσεις. Άρα θα πρέπει να κατασκευαστεί μια εικόνα του γράφου όπως αυτή ήταν στο παρελθόν. Χωρίς
βλάβη της γενικότητας, για το σκοπό αυτό κάνουμε την εξής θεώρηση: Στο γράφο με την πάροδο του χρόνο
δεν διαγράφονται συνδέσεις (ακμές) παρά μόνο δημιουργούνται νέες. Επομένως σε μια παρελθοντική στιγμή
απλώς θα υπήρχαν λιγότερες συνδέσεις, και προφανώς θα φτάσουμε στην εικόνα αυτή του γράφου διαγράφοντας
κάποιες ακμές με τυχαίο τρόπο.

Ξεκινώντας λοιπόν με μια εικόνα του γράφου στο παρελθόν, όπως αναφέρθηκε, με χρήση τεχνικών μηχανικής
μάθησης θα επιχειρήσουμε να προβλέψουμε, ή αλλιώς να ανακαλύψουμε τις νέες συνδέσεις που θα προκύψουν
μέχρι το παρόν. Για την επίτευξη του σκοπού αυτού, θα ακολουθηθεί η εξής λογική: Σκοπός είναι να 
προβλεφθούν νέες συνδέσεις, ή διαφορετικά, να προβλεφθεί ορθά αν μια σύνδεση (ακμή) είναι αληθής ή 
όχι. Συνεπώς το ζήτημα ανάγεται σε ένα πρόβλημα δυαδικής ταξινόμησης, και απαντάται το ερώτημα αν μια
ακμή υπάρχει (1) ή δεν υπάρχει (0). Οπότε θα χρησιμοποιηθεί ένας δυαδικός ταξινομητής. Για τη 
λειτουργία του ταξινομητή θα πρέπει να αναπαρασταθούν τα δεδομένα σε κατάλληλη μορφή, αφού η αρχική
μορφή του γράφου δεν είναι συμβατή με τέτοια συστήματα. Για την επίτευξη του σκοπού αυτού θα 
χρησιμοποιηθούν διάφορες τεχνικές αναπαράστασης γράφων. Κλείνοντας, να αναφερθεί ότι στην αναπαράσταση των δεδομένων 
χρησιμοποιούνται τεχνικές βαθειάς μάθησης ενώ όσον αφορά τους ταξινομητές χρησιμοποιούνται τεχνικές επιβλεπόμενης μάθησης.

\section{Δομή της εργασίας}

Η εργασία χωρίζεται σε τρείς ενότητες. Αρχικά θα δοθούν κάποια βασικά στοιχεία θεωρίας γράφων, ανάλυσης
δικτύων και κάποια στοιχεία σχετικά με τη μηχανική μάθηση. Θα παρουσιαστεί δηλαδή σύντομα όλο το θεωρητικό υπόβαθρο
πάνω στο οποίο βασίστηκε η εργασία αυτή. Επίσης, θα αναλυθούν σε βάθος οι τεχνικές αναπαράστασης 
γράφων. Στο δεύτερο σκέλος θα παρουσιαστεί η μεθοδολογία που ακολουθήθηκε καθώς και η δομή των πειραμάτων
ώστε να μπορεί ο αναγνώστης να αναπαράξει επακριβώς τα πειράματα,
ενώ στο τρίτο σκέλος θα δοθούν οι ακριβείς συνθήκες κάτω από τις οποίες πραγματοποιήθηκαν τα πειράματα,
καθώς και τα αποτελέσματα αυτών. Τέλος, θα δοθούν συγκεκριμένα συμπεράσματα και θα προταθούν 
ενδεχόμενες μελλοντικές επεκτάσεις που μπορούν να εξεταστούν.